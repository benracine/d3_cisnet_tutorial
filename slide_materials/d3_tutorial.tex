%  d3 Tutorial CISNET Programmers Breakout November 2, 2011
%  Author: Ben Racine
%  Date:   November 2, 2011

\documentclass{beamer}

\mode<presentation>
{
    \usetheme{Warsaw}
    \setbeamercovered{transparent}
}

\usepackage{amsmath,amssymb}
\usepackage{colortbl}
\usepackage[english]{babel}
\usepackage{fancybox, stmaryrd}
\usepackage[latin1]{inputenc}
\usepackage{lmodern}
\usepackage{pgf, pgfnodes}
\usepackage[T1]{fontenc} 
\usepackage{times}
\usepackage{verbatim}
\definecolor{links}{HTML}{2A1B81}
\hypersetup{colorlinks,linkcolor=,urlcolor=links}


\title{PBO Workshop} 
\subtitle{Data-Driven Documents With d3} 
\author{Ben Racine \inst{1} }
\institute{\inst{1} Cornerstone Systems NW }
\date{November 2, 2011}
\subject{Data Visualization}


\begin{document}


\begin{frame}
    \frametitle{}
    \titlepage
\end{frame}


\begin{frame}
    \frametitle{Outline}
    \tableofcontents
\end{frame}


\section{Introduction}

\subsection{Attendee Introduction}

\begin{frame}
    \frametitle{Introductions}
    \begin{block}{Favorite Tools?}
        \pause
        \begin{itemize}
        \item Database and/or spreadsheet tool
        \item DSL and/or general programming language
        \item Visualization tools
        \end{itemize}
    \end{block}
    \pause
    \begin{block}{Any web developers?}
        \begin{itemize}
        \item Javascript experience?
        \end{itemize}
    \end{block}
\end{frame}


\subsection{Browser Poll}

\begin{frame}
    \frametitle{Introductions}
    \begin{block}{Browser Usage}
        \begin{itemize}
        \item Chrome
        \pause
        \item Firefox 3+
        \pause
        \item IE9
        \pause
        \item Safari
        \pause
        \item Opera
        \pause
        \item None of the above?
        \end{itemize}
    \end{block}
\end{frame}


\section{Installation}


\begin{frame}[fragile]
    \frametitle{Introductions}
    \begin{block}{Decision Tree}
        \tiny{
        \begin{verbatim}
if you are comfortable developing in the browser
   if you are a git user
       git clone git@github.com:benracine/d3_cisnet_tutorial.git
   else
       https://github.com/benracine/d3_cisnet_tutorial/downloads
   end
else
    jsfiddle links
end
        \end{verbatim}
        }
    \end{block}
%   \footnote{Recommended to serve up on localhost on Chrome, i.e. python -m SimpleHTTPServer 8000}
    \footnote{jsfiddle examples still need to be done}
\end{frame}


\section{Resources}


\begin{frame}
    \frametitle{Resources}
        \begin{itemize}
        \item \href{http://mbostock.github.com/d3/}{Github}
            \begin{itemize}
            \item \href{https://github.com/mbostock/d3/wiki/API-Reference}{API}
            \item \href{http://mbostock.github.com/d3/ex/}{Examples}
            \item \href{}{Extended Examples}
            \item \href{https://github.com/mbostock/d3.git}{Source}
            \end{itemize}
        \item \href{http://groups.google.com/group/d3-js?pli=1}{Google group}
        \item \href{http://www.w3.org/TR/SVG/}{SVG Specification (v1.1)}
        \item Twitter: $@i3enhamin, @mbostock$
        \end{itemize}
\end{frame}


%\begin{frame}
%    \frametitle{Ready to Get Started?}
%    \begin{block}{Canonical Test}
%        \begin{itemize}
%        \item Navigate to an exercise file
%        \item Open up web developer tools
%        \item Go to console
%        \item Enter d3 and you should see 'object' in the response
%        \end{itemize}
%    \end{block}
%\end{frame}
%
%
%
%\section{Background}
%
%\begin{frame}
%    \frametitle{Background}
%    \begin{block}{jQuery + Protovis}
%        \begin{itemize}
%        \item Any jQuery experience?
%            \pause
%            \begin{itemize}
%            \item D3 is similar, but can also target the SVG (xml-like image format)
%            \pause
%            \item They both do some fancy functional programming to make it possible 
%                  for us to declaratively (and efficiently) reach into the dom tree.
%            \pause
%        \item Any Protovis exposure by any chance?
%       \end{itemize}
%    \end{block}
%\end{frame}


%\section{Tutorials}
%
%
%\begin{frame}
%    \frametitle{Tutorials}
%    \begin{block}{exercise-01.html}
%        \begin{itemize}
%        \item  Hello world
%        \item  Concepts
%        \begin{itemize}
%			\item sourcing d3 file 
%            \item unified d3 namepace
%            \item css3 selector notation
%            \item element, class, id, attribute, descendants, etc.
%            \item d3.select("") ~ \$("") ~ jQuery("")
%            \item cascading/callbacks
%       \end{itemize}
%    \end{block}
%    \footnote{We are only manipulating html elements so far}
%\end{frame}


%\begin{frame}
%    \frametitle{exercise-03.html}
%    \begin{block}{}
%        \begin{itemize}
%        \item 
%%\begin{frame}
%%		 	- get rid of text
%%		 	- get rid of clever tranform
%%		    .append("svg:svg")  --> mention that he has other namespace**
%%	    	.attr("width", w)
%%	    	make a namespace variable for the svg canvas (not to be confused with the "other" <canvas>)
%%	    		it's highly likely that we will want to reuse it and traversing the dom tree
%%	    		to get here is 'expensive'
%%\end{frame}
%        \item 
%        \end{itemize}
%    \end{block}
%\end{frame}
%
%\begin{frame}
%    \frametitle{}
%    \begin{block}{}
%        \begin{itemize}
%        \item 
%%\begin{frame}
%%		 exercise-10 ... 
%%		 	identity function**
%%		 		- functional programming
%%		 	data binding
%%		 		update
%%		 		enter
%%		 		exit
%%		 			explain the misnomer nature, stage/venn diagram analogy
%%		 	g namespace
%%		 	clever transform
%%\end{frame}
%        \item 
%        \end{itemize}
%    \end{block}
%\end{frame}
%
%
%\begin{frame}
%    \frametitle{}
%    \begin{block}{}
%        \begin{itemize}
%        \item 
%%		 exercise-11 ... 
%%		 	2D array into html table
%%		 	scales
%        \item 
%        \end{itemize}
%    \end{block}
%\end{frame}
%
%
%\begin{frame}
%    \frametitle{}
%    \begin{block}{}
%        \begin{itemize}
%        \item 
%%		 exercise-12 ... bar chart
%%		 	range bands**
%        \item 
%        \end{itemize}
%    \end{block}
%\end{frame}
%
%
%\begin{frame}
%    \frametitle{}
%    \begin{block}{}
%        \begin{itemize}
%%		 finale: exercise 13 bar chart with axes elements
%%		 	ex
%        \item 
%        \item 
%        \end{itemize}
%    \end{block}
%\end{frame}
%
%
%\begin{frame}
%    \frametitle{}
%    \begin{block}{}
%        \begin{itemize}
%%		 Extras
%%			 css-hover
%%			 d3 event-based transition
%%			 ease
%%			 tween
%%			 interpolate
%        \item 
%        \item 
%        \end{itemize}
%    \end{block}
%\end{frame}
%
%\begin{frame}
%    \frametitle{}
%    \begin{block}{}
%        \begin{itemize}
%        \item 
%%Conclusion / summary
%        \item 
%        \end{itemize}
%    \end{block}
%\end{frame}
%
%
%\begin{frame}
%    \frametitle{}
%    \begin{block}{}
%        \begin{itemize}
%%Checklist
%%	Ensure that 6 examples are as I want them
%%	5 + 6 slides
%        \item 
%        \item 
%        \end{itemize}
%    \end{block}
%\end{frame}
%
%** Perhaps we should videotape it
\end{document}

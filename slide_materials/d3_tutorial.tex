%  d3 Tutorial CISNET Programmers Breakout November 2, 2011
%  Author: Ben Racine
%  Date:   November 2, 2011
%

\documentclass{beamer}
\mode<presentation>
{
    \usetheme{Warsaw}
    \setbeamercovered{transparent}
}

\setbeamercolor{math text}{fg=red!50!black}

\usepackage{amsmath,amssymb}
\usepackage{colortbl}
\usepackage[english]{babel}
\usepackage{fancybox, stmaryrd}
\usepackage[latin1]{inputenc}
\usepackage{lmodern}
\usepackage{pgf, pgfnodes}
\usepackage[T1]{fontenc} 
\usepackage{times}
\usepackage{verbatim}

\definecolor{links}{HTML}{2A1B81}
\hypersetup{colorlinks,linkcolor=,urlcolor=links}


\title{PBO Workshop} 
\subtitle{Data-Driven Documents With d3} 
\author{Ben Racine \inst{1} }
\institute{\inst{1} Cornerstone Systems NW }
\date{November 2, 2011}
\subject{Data Visualization}


\begin{document}


\begin{frame}
    \frametitle{}
    \titlepage
\end{frame}


%\begin{frame}
%    \scriptsize{
%        \frametitle{Outline}
%        \tableofcontents[pausesections]
%    }
%\end{frame}



\section{Introduction}

\subsection{Attendee Introduction}

\begin{frame}
    \frametitle{Attendee Introduction}
    \begin{itemize}
    These materials live at http://mbostock.github.com/d3/}{http://mbostock.github.com/d3/}
    \end{itemize}
\end{frame}

\begin{frame}
    \frametitle{Attendee Introduction}
\pause
    \begin{block}{Favorite Tools?} 
\pause
    \begin{itemize}
    \item Database and/or spreadsheet tools
\pause
    \item DSL and/or general programming languages
\pause
    \item Visualization tools
    \end{itemize}
    \end{block}
\pause
    \begin{block}{Any web developers?}
        \begin{itemize}
\pause
        \item Javascript experience?
        \end{itemize}
    \end{block}
\end{frame}


\subsection{Browser Poll}

\begin{frame}
    \frametitle{Introductions}
\pause
    \begin{block}{Browser Poll}
        \begin{itemize}
\pause
        \item Chrome
\pause
        \item Firefox 3+
\pause
        \item IE9
\pause
        \item Safari
\pause
        \item None of the above?
        \end{itemize}
    \end{block}
\end{frame}


\section{Background}

\begin{frame}
    \frametitle{Background}
    \begin{block}{jQuery $+$ Protovis $\approx$ d3}
\pause
        \begin{itemize}
        \item Any jQuery experience?
\pause
            \begin{itemize}
            \item d3 is similar, but can also target the SVG (xml-like image format)
\pause
            \item They both do some fancy functional programming to make it possible 
                  for us to declaratively (and efficiently) reach into the dom tree
            \end{itemize}
\pause
       \item Any Protovis exposure by any chance?
       \end{itemize}
    \end{block}
\end{frame}


\section{Resources}

\begin{frame}
    \frametitle{Resources}
\pause
        \begin{itemize}
        \item \href{http://mbostock.github.com/d3/}{\underline{Github}: http://mbostock.github.com/d3/}
\pause
            \begin{itemize}
            \item \href{https://github.com/mbostock/d3/wiki/API-Reference}{\underline{API}: https://github.com/mbostock/d3/wiki/API-Reference}
\pause
            \item \href{http://mbostock.github.com/d3/ex/}{\underline{Examples}: http://mbostock.github.com/d3/ex/}
\pause
            \item \href{https://github.com/mbostock/d3.git}{\underline{Source}: https://github.com/mbostock/d3.git}
            \end{itemize}
\pause
        \item \href{http://groups.google.com/group/d3-js?pli=1}{Google message group}
\pause
        \item \href{http://www.w3.org/TR/SVG/}{SVG Specification (v1.1)}
\pause
        \item Twitter: $@i3enhamin, @mbostock$
        \end{itemize}
\end{frame}



\section{Installation}

\subsection{Clone or Download Slides, Source Code and Exercises}

\begin{frame}[fragile]
    \frametitle{Clone or Download Slides, Source Code and Exercises}
        \tiny{
        \begin{verbatim}
   if you are a git user
       git clone git@github.com:benracine/d3_cisnet_tutorial.git
   else
       https://github.com/benracine/d3_cisnet_tutorial/downloads
   end
        \end{verbatim}
        }
\end{frame}



\subsection{Canonical Test to Ensure Installation}

\begin{frame}
    \frametitle{Canonical Test to Ensure Installation}
    \begin{itemize}
\pause
    \item Navigate to an exercise file on your hard-drive in your browser
\pause
    \item Open up your browser's web developer tools
\pause
    \item Go to the developer tool console
\pause
    \item Enter d3 and you should see "object" in the response
\pause
    \end{itemize}
\end{frame}




\section{Tutorials: Round One}

\subsection{Briefly playing in the console}

\begin{frame}
    \frametitle{Briefly playing in the console}
    \begin{itemize}
    \item Navigate to \href{http://mbostock.github.com/d3/}{http://mbostock.github.com/d3/}
    \item Let's change the color of the hyperlinks
        \begin{itemize}
        \item Open console
        \item d3.selectAll("a").style("color","red")
        \item d3.selectAll("p").style("color","blue")
        \end{itemize}
    \item Note the existence of both d3.select and d3.selectAll
    \end{itemize}
\end{frame}



\subsection{Hello world}

\begin{frame}
    \frametitle{Exercise-01.html: Hello World}
    \begin{itemize}
    \item This example only uses raw html (i.e. no SVG)
    \item Include the main d3 file in line 5
        \begin{itemize}
        \item This, d3.js, is the 'core' module
        \item The default build of d3.js includes the core, scale, svg and behavior modules
        \item Others include:
            \begin{itemize}
            \item d3.time.js
            \item d3.geo.js
            \item d3.csv.js
            \end{itemize}
        \end{itemize}
    \end{itemize}
\end{frame}


\begin{frame}
    \frametitle{Exercise-01.html: Hello World}
    \begin{itemize}
    \item All d3 commands live in a unified d3 namespace
    \item d3 supports CSS3 notation, i.e. one can select by:
        \begin{itemize}
        \item Tag ("div")
        \item Class (".awesome")
        \item Identifier ("$#foo$")
        \item Containment ("parent child")
        \item Intersection (".this.that" for logical AND) 
        \item Union (".this, .that" for logical OR)
        \item Attribute ("$[color=red]$")
        \end{itemize}
    \item Notice that method chaining has already begun
    \item Method chaining = returning the same selection for further modification
    \item Elements can be accessed directly
        \begin{itemize}
        \item (e.g., selection$[0][0]$)
        \item Through the each call
        \end{itemize}
    \end{itemize}
\end{frame}



\begin{frame}
    \frametitle{Exercise-01.html: Hello World}
    \begin{itemize}
    \item .text() is an "operator"
    \item Operators can both get or set:
        \begin{itemize}
        \item .classed() : toggling of css classes
        \item .style() : sets the CSS style property (can be run w/ priority levels)
        \item .property() : example, a slider value
        \item .property() : example, a slider value
        \end{itemize}
    \item By default, D3 supports svg, xhtml, xlink, xml and xmlns namespaces
    \item Additional namespaces can be registered
    \end{itemize}
\end{frame}


\begin{frame}
    \frametitle{Exercise-01.html: Hello World}
    \begin{itemize}
    \item Can be set as either constants or as functions
    \item When used to set document content, the operators return the current selection, so you can chain multiple operators together in a concise statement.
    \item d3.select("") $\approx$ \$("") $\approx$ jQuery("")
    \end{itemize}
\end{frame}




\subsection{Including an SVG element}

\begin{frame}
    \frametitle{Exercise-02.html:: Including an SVG Element}
    \begin{itemize}
    \item Width and height could be related to the width and height of the window
    \item Think of the svg element as a canvas with a transformed coordinate system
    \item A svg:g element is means of containing other svg elements
    \item A tranform can be a handy way of moving the coordinate system to a desired location
    \item Regarding the coordinate system, note:
        \begin{itemize}
        \item Origin is the top-left
        \item x is positive to the right
        \item y is positive down
        \item scales can be used to correct to cartesian coords
        \end{itemize}
    \end{itemize}
\end{frame}


\begin{frame}
    \frametitle{Exercise-02.html:: Including an SVG Element}
    \begin{itemize}
    \item svg:circle self explanatory
        \begin{itemize}
        \item Refer to the SVG spec for relevant and/or required circle attributes
        \end{itemize}
    \item Note the use of a JavaScript namespace variable to cache a selection of interest
    \end{itemize}
\end{frame}



\subsection{Combining with CSS Selections}

\begin{frame}
    \frametitle{Exercise-03.html: Combining with CSS Selections}
        \begin{itemize}
        \item Concepts
            \begin{itemize}
            \item CSS3 selector notation in the style section $\approx$ in the d3.select("") command
            \item Appending is fairly self-explanatory
            \item Good practice to use intelligent id and class attributes
            \end{itemize}
        \end{itemize}
\end{frame}


\begin{frame}
    \frametitle{Exercise-03.html: Combining with CSS Selections}
        \begin{itemize}
                \item Namespaces, explain that svg:svg <-- first one is a namespace, second one is the element itself svg:g is kind of like a div in html:... just a bag in which to group other things in note: you give them uniqueness through class or id
                \item Attr, addressed in previous slide
                \item Appropriate use of namespace variables
                \item Assign a namespace at any "juncture" in your workflow i.e. if you're about to add circles AND text to your scenegraph... it's probably appropriate to add a name to the state of your scenegraph at that point
            \end{itemize}
        \end{itemize}
\end{frame}



\subsection{Event Listeners}

\begin{frame}
    \frametitle{Exercises-05.html through Exercise-08.html: Skipping for now}
    \begin{itemize}
    \item d, i, and this
    \item Event listeners can take many forms
    \item Can listen for different types of events
    \item Click, mouseover, submit, etc.
    \item There's a subtlety of attaching to multiple functions to the same event...
    \item i.e. click.foo maps to one function, click.bar maps to another function
    \end{itemize}
\end{frame}



\subsection{Tweens, Scaling, User-events}

\begin{frame}
    \frametitle{Exercises-05.html through Exercise-08.html: Skipping for now}
    \begin{itemize}
    \item exercise-05.html: skip tweens and get to data bindings
    \item exercise-06.html: notice that we're scaling the whole image,
    \item exercise-07.html: listen to user events, i.e watch the mouse move
    \item exercise-08.html: mouse fading events
    \item exercise-09.html: html-based bar-chart to emphasize that it's not just for SVG canvases
    \end{itemize}
\end{frame}



\section{Tutorials: Round Two}

\subsection{Bar Chart}

\begin{frame}
    \frametitle{Exercise-09.html: Bar Chart}
    \begin{itemize}
    \item Bar Chart with HTML Elements
    \item Scales
    \end{itemize}
\end{frame}

\begin{frame}
    \frametitle{Exercise-09.html: Bar Chart}
    \begin{itemize}
    \item Identity function
    \item Functional programming
    \item Data binding selections
    \item Update
    \item Enter
    \item Exit
    \end{itemize}
\end{frame}



\subsection{2d Array into HTML Table}

\begin{frame}
    \frametitle{Exercise-11.html: 2d Array into an HTML Table}
    \begin{itemize}
    \item Foo
    \end{itemize}
\end{frame}



\subsection{2d Array into SVG Bar Chart}

\begin{frame}
    \frametitle{Exercise-12.html: 2d Array into SVG Bar Chart}
    \begin{itemize}
    \item 2d Array into SVG Bar Chart
    \item RangeBands
    \item Linear vs. ordinal scales
    \end{itemize}
\end{frame}



\subsection{Axes Elements}

\begin{frame}
    \frametitle{Exercise-13.html: Axes Elements}
    \begin{itemize}
    \item 
    \end{itemize} 
\end{frame}



\subsection{Extras}

\begin{frame}
    \frametitle{Extras}
    \begin{itemize}
    \item Transition $\approx$ a non-instantaneous transformation with extra attributes:
        \begin{itemize}
        \item Duration
        \item Delay
        \end{itemize}
    \item Ease
    \item Interpolate
    \item Tween (exercise-05.html if we get a chance)
    \item Call and each for control flow 
    \end{itemize}
\end{frame}



\section{Conclusion}

\begin{frame}
    \frametitle{Conclusion}
    \begin{itemize}
    \item 
    \end{itemize}
\end{frame}



\end{document}
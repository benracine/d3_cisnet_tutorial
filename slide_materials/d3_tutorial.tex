%  d3 Tutorial CISNET Programmers Breakout November 2, 2011
%  Author: Ben Racine
%  Date:   November 2, 2011

\documentclass{beamer}

\mode<presentation>
{
    \usetheme{Warsaw}
    \setbeamercovered{transparent}
}

\usepackage{amsmath,amssymb}
\usepackage{colortbl}
\usepackage[english]{babel}
\usepackage{fancybox, stmaryrd}
\usepackage[latin1]{inputenc}
\usepackage{lmodern}
\usepackage{pgf, pgfarrows, pgfnodes}
\usepackage[T1]{fontenc} 
\usepackage{times}
\usepackage{verbatim}
\definecolor{links}{HTML}{2A1B81}
\hypersetup{colorlinks,linkcolor=,urlcolor=links}


\title{PBO Workshop} 
\subtitle{Data-Driven Documents With d3} 
\author{Ben Racine \inst{1} }
\institute{\inst{1} Cornerstone Systems NW }
\date{November 2, 2011}
\subject{Data Visualization}


\begin{document}


\begin{frame}
    \frametitle{}
    \titlepage
\end{frame}


\begin{frame}
    \frametitle{Outline}
    \tableofcontents
\end{frame}


\section{Introduction}

\subsection{Attendee Introduction}

\begin{frame}
    \frametitle{Introductions}
    \begin{block}{Favorite Tools?}
        \pause
        \begin{itemize}
        \item Database and/or spreadsheet tool
        \item DSL and/or general programming language
        \item Visualization tools
        \end{itemize}
    \end{block}
    \pause
    \begin{block}{Any web developers?}
        \begin{itemize}
        \item Javascript experience?
        \end{itemize}
    \end{block}
\end{frame}


\subsection{Browser Poll}

\begin{frame}
    \frametitle{Introductions}
    \begin{block}{Browser Usage}
        \begin{itemize}
        \item Chrome
        \pause
        \item Firefox 3+
        \pause
        \item IE9
        \pause
        \item Safari
        \pause
        \item Opera
        \pause
        \item None of the above?
        \end{itemize}
    \end{block}
\end{frame}


\section{Installation}

\begin{frame}[fragile]
    \frametitle{Introductions}
    \begin{block}{Decision Tree}
        \begin{verbatim}
if you have developed in the browser
   if you are a git user
       git clone git@github.com:benracine/d3_cisnet_tutorial.git
   else if you have a dropbox account
       dropbox.com/adfasdfasdfasdf
   else
       a thumb drive is floating around
   end
else
    jsfiddle links 1n
end
        \end{verbatim}
    \end{block}
    \footnote{Chrome might have funny behavior if you don't serve up d3 root as localhost}
\end{frame}


\begin{frame}
    \frametitle{Ready to Get Started?}
    \begin{block}{Canonical Test}
        \begin{itemize}
        \item Open up web developer tools
        \item Go to console
        \item Enter d3 and you should see 'object' as the response
        \end{itemize}
    \end{block}
\end{frame}


%\begin{frame}
%    \frametitle{Resources}
%        %\begin{itemize}
%        %\item \href{http://mbostock.github.com/d3/}{Github}
%        %    \begin{itemize}
%        %    \item \href{https://github.com/mbostock/d3/wiki/API-Reference}{API}
%        %    \item \href{http://mbostock.github.com/d3/ex/}{Examples}
%        %    \item \href{}{Extended Examples}
%        %    \item \href{https://github.com/mbostock/d3.git}{Source}
%        %    \end{itemize}
%        %\item \href{}{Online tutorials}
%        %\item \href{}{Google group}
%        %\item \href{}{SVG Specification (v1.1)}
%        %\item Twitter: @i3enhamin, @mbostock
%        %\end{itemize}
%\end{frame}
%
%
%\begin{frame}
%    \frametitle{}
%    \begin{block}{}
%        \begin{itemize}
%        \item 
%        \item 
%        \end{itemize}
%    \end{block}
%\end{frame}
%
%%Tutorials
%%	Ok, if you're not here... I apologize, but I strongly encourage you to stay
%%	so that you can absorb the spirit and go play in your free time.
%%	Discuss the tutorials/topic outline
%%	Who has jQuery experience? D3 is similar, but can also target SVG -> xml-like image format
%%		- Long story short... they do some fancy functional programming
%%		  to make it possible for us to declaratively (and efficiently)
%%		  reach into the dom tree.
%%
%\begin{frame}
%    \frametitle{}
%    \begin{block}{}
%        \begin{itemize}
%        \item 
%        \item 
%        \end{itemize}
%    \end{block}
%\end{frame}
%
%\begin{frame}
%    \frametitle{}
%    \begin{block}{}
%        \begin{itemize}
%%        \item 
%%		 exercise-01 ... modify per our use
%%			 source d3 file 
%%			 unified d3 namepace
%%			 selector notation
%%				 supports css3 selector notation
%%					 element, class, id, attribute
%%					 descendant
%%					 etc**
%%				 d3.select("") ~ $("") ~ jQuery("")
%%			 cascading/callbacks
%%			 note that we are only manipulating html elements so far
%        \item 
%        \end{itemize}
%    \end{block}
%\end{frame}
%
%\begin{frame}
%    \frametitle{}
%    \begin{block}{}
%        \begin{itemize}
%        \item 
%%\begin{frame}
%%		 exercise-03 
%%		 	- get rid of text
%%		 	- get rid of clever tranform
%%		    .append("svg:svg")  --> mention that he has other namespace**
%%	    	.attr("width", w)
%%	    	make a namespace variable for the svg canvas (not to be confused with the "other" <canvas>)
%%	    		it's highly likely that we will want to reuse it and traversing the dom tree
%%	    		to get here is 'expensive'
%%\end{frame}
%        \item 
%        \end{itemize}
%    \end{block}
%\end{frame}
%
%\begin{frame}
%    \frametitle{}
%    \begin{block}{}
%        \begin{itemize}
%        \item 
%%\begin{frame}
%%		 exercise-10 ... 
%%		 	identity function**
%%		 		- functional programming
%%		 	data binding
%%		 		update
%%		 		enter
%%		 		exit
%%		 			explain the misnomer nature, stage/venn diagram analogy
%%		 	g namespace
%%		 	clever transform
%%\end{frame}
%        \item 
%        \end{itemize}
%    \end{block}
%\end{frame}
%
%
%\begin{frame}
%    \frametitle{}
%    \begin{block}{}
%        \begin{itemize}
%        \item 
%%		 exercise-11 ... 
%%		 	2D array into html table
%%		 	scales
%        \item 
%        \end{itemize}
%    \end{block}
%\end{frame}
%
%
%\begin{frame}
%    \frametitle{}
%    \begin{block}{}
%        \begin{itemize}
%        \item 
%%		 exercise-12 ... bar chart
%%		 	range bands**
%        \item 
%        \end{itemize}
%    \end{block}
%\end{frame}
%
%
%\begin{frame}
%    \frametitle{}
%    \begin{block}{}
%        \begin{itemize}
%%		 finale: exercise 13 bar chart with axes elements
%%		 	ex
%        \item 
%        \item 
%        \end{itemize}
%    \end{block}
%\end{frame}
%
%
%\begin{frame}
%    \frametitle{}
%    \begin{block}{}
%        \begin{itemize}
%%		 Extras
%%			 css-hover
%%			 d3 event-based transition
%%			 ease
%%			 tween
%%			 interpolate
%        \item 
%        \item 
%        \end{itemize}
%    \end{block}
%\end{frame}
%
%\begin{frame}
%    \frametitle{}
%    \begin{block}{}
%        \begin{itemize}
%        \item 
%%Conclusion / summary
%        \item 
%        \end{itemize}
%    \end{block}
%\end{frame}
%
%
%\begin{frame}
%    \frametitle{}
%    \begin{block}{}
%        \begin{itemize}
%%Checklist
%%	Ensure that 6 examples are as I want them
%%	5 + 6 slides
%        \item 
%        \item 
%        \end{itemize}
%    \end{block}
%\end{frame}
%
%** Perhaps we should videotape it
\end{document}

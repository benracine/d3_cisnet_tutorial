%  d3 Tutorial CISNET Programmers Breakout November 2, 2011
%  Author: Ben Racine
%  Date:   November 2, 2011

\documentclass{beamer}
\mode<presentation>
{
    \usetheme{Warsaw}
    \setbeamercovered{transparent}
}

\setbeamercolor{math text}{fg=red!50!black}

\usepackage{amsmath,amssymb}
\usepackage{colortbl}
\usepackage[english]{babel}
\usepackage{fancybox, stmaryrd}
\usepackage[latin1]{inputenc}
\usepackage{lmodern}
\usepackage{pgf, pgfnodes}
\usepackage[T1]{fontenc} 
\usepackage{times}
\usepackage{verbatim}

\definecolor{links}{html}{2A1B81}
\hypersetup{colorlinks,linkcolor=,urlcolor=links}


\title{PBO Workshop} 
\subtitle{Data-Driven Documents With d3} 
\author{Ben Racine \inst{1} }
\institute{\inst{1} Cornerstone Systems NW }
\date{November 2, 2011}
\subject{Data Visualization}


\begin{document}


\begin{frame}
    \frametitle{}
    \titlepage
\end{frame}


\begin{frame}
    \scriptsize{
        \frametitle{Outline}
        \tableofcontents[pausesections]
    }
\end{frame}



\section{Introduction}

\subsection{Attendee Introduction}

\begin{frame}
    \frametitle{Introductions}
\pause
    \begin{block}{Favorite Tools?} 
    \begin{itemize}
        \item Database and/or spreadsheet tools
\pause
        \item DSL and/or general programming languages
\pause
        \item Visualization tools
        \end{itemize}
    \end{block}
\pause
    \begin{block}{Any web developers?}
        \begin{itemize}
        \item Javascript experience?
        \end{itemize}
    \end{block}
\end{frame}


\subsection{Browser Poll}

\begin{frame}
    \frametitle{Introductions}
    \begin{block}{Browser Poll}
        \begin{itemize}
\pause
        \item Chrome
\pause
        \item Firefox 3+
\pause
        \item IE9
\pause
        \item Safari
\pause
        \item None of the above?
        \end{itemize}
    \end{block}
\end{frame}


\section{Background}

\begin{frame}
    \frametitle{Background}
    \begin{block}{jQuery $+$ Protovis $\approx$ d3}
        \begin{itemize}
        \item Any jQuery experience?
\pause
            \begin{itemize}
            \item D3 is similar, but can also target the SVG (xml-like image format)
\pause
            \item They both do some fancy functional programming to make it possible 
                  for us to declaratively (and efficiently) reach into the dom tree.
            \end{itemize}
\pause
        \item Any Protovis exposure by any chance?
       \end{itemize}
    \end{block}
\end{frame}


\section{Resources}

\begin{frame}
    \frametitle{Resources}
\pause
        \begin{itemize}
        \item \href{http://mbostock.github.com/d3/}{\underline{Github}: http://mbostock.github.com/d3/}
            \begin{itemize}
            \item \href{https://github.com/mbostock/d3/wiki/API-Reference}{\underline{API}: https://github.com/mbostock/d3/wiki/API-Reference}
            \item \href{http://mbostock.github.com/d3/ex/}{\underline{Examples}: http://mbostock.github.com/d3/ex/}
            \item \href{https://github.com/mbostock/d3.git}{\underline{Source}: https://github.com/mbostock/d3.git}
            \end{itemize}
\pause
        \item \href{http://groups.google.com/group/d3-js?pli=1}{Google group}
\pause
        \item \href{http://www.w3.org/TR/SVG/}{SVG Specification (v1.1)}
\pause
        \item Twitter: $@i3enhamin, @mbostock$
        \end{itemize}
\end{frame}



\section{Installation}

\begin{frame}[fragile]
    \frametitle{Installation}
        \tiny{
        \begin{verbatim}
if you are comfortable developing in the browser
   if you are a git user
       git clone git@github.com:benracine/d3_cisnet_tutorial.git
   else
       https://github.com/benracine/d3_cisnet_tutorial/downloads
   end
else
    jsfiddle links
end
        \end{verbatim}
        }
    \footnote{Recommended to serve up on localhost on Chrome, i.e. python -m SimpleHTTPServer 8000}
    \footnote{jsfiddle examples still need to be done}
\end{frame}


\section{Tutorials}


\subsection{Canonical Test}

\begin{frame}
    \frametitle{Ready to Get Started?}
    \begin{block}{Canonical Test}
        \begin{itemize}
        \item Navigate to an exercise file on your hard-drive in your browser
        \item Open up your browser's web developer tools
        \item Go to console
        \item Enter d3 and you should see "object" in the response
        \end{itemize}
    \end{block}
\end{frame}


\subsection{Briefly playing in the console}

\begin{frame}
    \frametitle{Getting Started}
    \begin{itemize}
    \item navigate to \href{http://mbostock.github.com/d3/}{\underline{Github}: http://mbostock.github.com/d3/}
    \item let's change the color of the hyperlinks
        \begin{itemize}
        \item open console; d3.selectAll("a").style("color","red")
        \end{itemize}
    \end{itemize}


\subsection{Hello world}

\begin{frame}
    \frametitle{Exercise-01.html: Hello World}
    \begin{itemize}
    \item just raw html (i.e. no SVG)
    \item include the main d3 file in line 5
        \begin{itemize}
        \item this is the 'core' module
        \item the default build of d3.js includes the core, scale, svg and behavior modules
        \item others include:
            \begin{itemize}
            \item time
            \item geo
            \item csv
            \end{itemize}
        \end{itemize}
    \end{itemize}
\end{frame}


\begin{frame}
    \frametitle{Exercise-01.html: Hello World}
    \begin{itemize}
    \item all d3 commands live in a unified d3 namespace
    \item d3 supports CSS3 notation, i.e. can select by:
        \begin{itemize}
        \item can select by tag ("div")
        \item class (".awesome")
        \item unique identifier ("#foo")
        \item containment ("parent child")
        \item selectors can be intersected (".this.that" for logical AND) or unioned (".this, .that" for logical OR)
        \item attribute ("[color=red]")
        \end{itemize}
    \item note the difference between d3.select and d3.selectAll
    \item notice the method chaining has already begun
    \item elements can be accessed directly
        \begin{itemize}
        \item (e.g., selection[0][0])
        \item through the each call
        \end{itemize}
    \end{itemize}
\end{frame}


\begin{frame}
    \frametitle{Exercise-01.html: Hello World}
        \begin{itemize}
            \item text is an "operator"
            \item operators can both get or set 
            \item class: toggling of css classes
            \item style: sets the css style property, can be run w/ priority levels
            \item property: example, a slider value
            \item by default, D3 supports svg, xhtml, xlink, xml and xmlns namespaces. Additional namespaces can be registered by adding to d3.ns.prefix.
            \item can be set as either constants or as functions
            \item when used to set document content, the operators return the current selection, so you can chain multiple operators together in a concise statement.
            \item d3.select("") $\approx$ \$("") $\approx$ jQuery("")
        \end{itemize}
    \end{block}
\end{frame}


\subsection{Including an SVG element}

\begin{frame}
    \frametitle{Exercise-02.html:: Including an SVG Element}
    \begin{itemize}
    \item width and height could be related to the width and height of the window in order to be self-adjusting
    \item think of the svg element as a canvas with a transformed coordinate system
    \item g element is means of containing other svg elements
    \item tranform is a handy way of moving the coordinate system to a desired location
        \begin{itemize}
        \item coord system
        \item origin is the top-left
        \item x is positive to the right
        \item y is positive down
        \end{itemize}
    \item svg:circle self explanatory
        \begin{itemize}
        \item refer to the SVG spec for relevant circle attributes
        \end{itemize}
    \item note the use of the svg namespace variable to cache a selection of interest
    \end{itemize}
\end{frame}



\subsection{Combining with CSS Selections}

\begin{frame}
    \frametitle{Exercise-03.html: Combining with CSS Selections}
        \begin{itemize}
            \item Concepts
            \begin{itemize}
                \item CSS3 selector notation in the style section $\approx$ in the d3.select("") command
                \item Appending is fairly self-explanatory
                \item Transforms, coordinate transform to make it easier to think of
                \item Note that the origin is in the top-left corner and that positive y is down.
                \item Jumping ahead, I will commonly account for this, not with a transform, but with an appropriate scale, which we will get to shortly
                \item Note: a side effect of this transform is that by not setting the x,y variables on the circle... it defaults to zero.  This zero is now the point that is transformed to the middle of the page though.
                \item Namespaces, explain that svg:svg <-- first one is a namespace, second one is the element itself svg:g is kind of like a div in html:... just a bag in which to group other things in note: you give them uniqueness through class or id
                \item Attr, addressed in previous slide
                \item Appropriate use of namespace variables
                \item Assign a namespace at any "juncture" in your workflow i.e. if you're about to add circles AND text to your scenegraph... it's probably appropriate to add a name to the state of your scenegraph at that point
                \item 
            \end{itemize}
        \end{itemize}
\end{frame}



\subsection{Event Listeners}

\begin{frame}
    \frametitle{Exercises-05.html through Exercise-08.html: Skipping for now}
        \begin{itemize}
        \item d, i, this...
        \item event listeners can take many forms
        \item you can listen for different types of events
        \item click, mouseover, submit, etc.
        \item there's a subtlety of attaching to multiple functions to the same event...
        \item i.e. click.foo -> one function, click.bar -> another function
        \end{itemize}
\end{frame}



\subsection{Tweens, scaling, user-events}

\begin{frame}
    \frametitle{Exercises-05.html through Exercise-08.html: Skipping for now}
        \begin{itemize}
        \item exercise-05.html: skip tweens and get to data bindings
        \item exercise-06.html: notice that we're scaling the whole image,
        \item exercise-07.html: listen to user events, i.e watch the mouse move
        \item exercise-08.html: mouse fading events
        \item exercise-09.html: html-based bar-chart to emphasize that it's not just for SVG canvases
        \end{itemize}
\end{frame}



\subsection{Bar Chart}

\begin{frame}
    \frametitle{Tutorials}
    \begin{block}{exercise-10.html:}
        \begin{itemize}
        \item Bar Chart with html: Elements
        \item have to explain scales... 
        \item Concepts
        \item exercise-09.html:
        \item how to draw a bar graph
        \item the basics of appending 
        \item Identity function
        \item Functional programming
        \item Data binding selections
        \item Update
        \item Enter
        \item Exit
        \end{itemize}
    \end{block}
\end{frame}



\subsection{2d Array into HTML Table}

\begin{frame}
    \frametitle{Exercise-11.html: 2d Array into HTML Table}
        \begin{itemize}
        \end{itemize}
    \end{block}
\end{frame}



\subsection{2d Array into SVG Bar Chart}

\begin{frame}
    \frametitle{Exercise-12.html: 2d Array into SVG Bar Chart}
        \begin{itemize}
        \item 2d Array into SVG Bar Chart
        \item he's used rangebands and ordinal scales
        \item RangeBands
        \item Linear vs. ordinal scales
        \end{itemize}
    \end{block}
\end{frame}


\subsection{Axes Elements}
axes elements are the stars
do a quick little demo of changing the results to show off how flexible it is
\begin{frame}
    \frametitle{Tutorials}
exercise-13.html:
        \begin{itemize}
            
            \item Axes Elements
        \end{itemize} 
\end{frame}


%% exercise 14, 15, 16 are largely a wash for now


\begin{frame}
    \frametitle{Extras}
        \begin{itemize}
        \item Transition $\approx$ a non-instantaneous transformation with extra attributes:
            \begin{itemize}
            \item Duration
            \item Delay
            \end{itemize}
        \item Ease
        \item Interpolate
        \item Tween (exercise-05.html if we get a chance)
        \item Call and each for control flow 
        \end{itemize}
    \end{block}
\end{frame}


\section{Conclusion}

\begin{frame}
    \frametitle{Conclusion}
    \begin{itemize}
        \item Blah
    \end{itemize}
\end{frame}

\end{document}
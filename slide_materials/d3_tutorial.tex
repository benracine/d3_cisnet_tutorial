%  d3 Tutorial CISNET Programmers Breakout November 2, 2011
%  Author: Ben Racine
%  Date:   November 2, 2011

\documentclass{beamer}
\mode<presentation>
{
    \usetheme{Warsaw}
    \setbeamercovered{transparent}
}

\setbeamercolor{math text}{fg=red!50!black}

\usepackage{amsmath,amssymb}
\usepackage{colortbl}
\usepackage[english]{babel}
\usepackage{fancybox, stmaryrd}
\usepackage[latin1]{inputenc}
\usepackage{lmodern}
\usepackage{pgf, pgfnodes}
\usepackage[T1]{fontenc} 
\usepackage{times}
\usepackage{verbatim}

\definecolor{links}{HTML}{2A1B81}
\hypersetup{colorlinks,linkcolor=,urlcolor=links}


\title{PBO Workshop} 
\subtitle{Data-Driven Documents With d3} 
\author{Ben Racine \inst{1} }
\institute{\inst{1} Cornerstone Systems NW }
\date{November 2, 2011}
\subject{Data Visualization}


\begin{document}


\begin{frame}
    \frametitle{}
    \titlepage
\end{frame}


\begin{frame}
    \scriptsize{
        \frametitle{Outline}
        \tableofcontents[pausesections]
    }
\end{frame}



\section{Introduction}

\subsection{Attendee Introduction}

\begin{frame}
    \frametitle{Introductions}
    \pause
    \begin{block}{Favorite Tools?} 
    \begin{itemize}
        \item Database and/or spreadsheet tools
        \pause
        \item DSL and/or general programming languages
        \pause
        \item Visualization tools
        \end{itemize}
    \end{block}
    \pause
    \begin{block}{Any web developers?}
        \begin{itemize}
        \item Javascript experience?
        \end{itemize}
    \end{block}
\end{frame}


\subsection{Browser Poll}

\begin{frame}
    \frametitle{Introductions}
    \begin{block}{Browser Poll}
        \begin{itemize}
        \pause
        \item Chrome
        \pause
        \item Firefox 3+
        \pause
        \item IE9
        \pause
        \item Safari
        \pause
        \item None of the above?
        \end{itemize}
    \end{block}
\end{frame}


\section{Background}

\begin{frame}
    \frametitle{Background}
    \begin{block}{jQuery $+$ Protovis $\approx$ d3}
        \begin{itemize}
        \pause
        \item Any jQuery experience?
        \pause
            \begin{itemize}
            \item D3 is similar, but can also target the SVG (xml-like image format)
        \pause
            \item They both do some fancy functional programming to make it possible 
                  for us to declaratively (and efficiently) reach into the dom tree.
            \end{itemize}
        \pause
        \item Any Protovis exposure by any chance?
       \end{itemize}
    \end{block}
\end{frame}


\section{Resources}

\begin{frame}
    \frametitle{Resources}
        \pause
        \begin{itemize}
        \item \href{http://mbostock.github.com/d3/}{\underline{Github}: http://mbostock.github.com/d3/}
            \begin{itemize}
            \item \href{https://github.com/mbostock/d3/wiki/API-Reference}{\underline{API}: https://github.com/mbostock/d3/wiki/API-Reference}
            \item \href{http://mbostock.github.com/d3/ex/}{\underline{Examples}: http://mbostock.github.com/d3/ex/}
            \item \href{https://github.com/mbostock/d3.git}{\underline{Source}: https://github.com/mbostock/d3.git}
            \end{itemize}
        \pause
        \item \href{http://groups.google.com/group/d3-js?pli=1}{Google group}
        \pause
        \item \href{http://www.w3.org/TR/SVG/}{SVG Specification (v1.1)}
        \pause
        \item Twitter: $@i3enhamin, @mbostock$
        \end{itemize}
\end{frame}



\section{Installation}

\begin{frame}[fragile]
    \frametitle{Installation}
        \tiny{
        \begin{verbatim}
if you are comfortable developing in the browser
   if you are a git user
       git clone git@github.com:benracine/d3_cisnet_tutorial.git
   else
       https://github.com/benracine/d3_cisnet_tutorial/downloads
   end
else
    jsfiddle links
end
        \end{verbatim}
        }
    \footnote{Recommended to serve up on localhost on Chrome, i.e. python -m SimpleHTTPServer 8000}
    \footnote{jsfiddle examples still need to be done}
\end{frame}


\section{Tutorials}


\subsection{Canonical Test}

\begin{frame}
    \frametitle{Ready to Get Started?}
    \begin{block}{Canonical Test}
        \begin{itemize}
        \pause
        \item Navigate to an exercise file on your hard-drive in your browser
        \pause
        \item Open up your browser's web developer tools
        \pause
        \item Go to console
        \pause
        \item Enter d3 and you should see "object" in the response
        \end{itemize}
    \end{block}
\end{frame}


\subsection{Briefly playing in the console}

\begin{frame}
    \frametitle{Getting Started}
    \begin{itemize}
    \item navigate to \href{http://mbostock.github.com/d3/}{\underline{Github}: http://mbostock.github.com/d3/}
    \item let's change the color of the hyperlinks
        \begin{itemize}
        \item open console; d3.selectAll("a").style("color","red")
        \end{itemize}
    \end{itemize}
\subsection{Hello World}



\subsection{Hello world}

\begin{frame}
    \frametitle{Exercise-01.html}
    \begin{itemize}
    \item just raw HTML (i.e. no SVG)
    \item include the main d3 file in line 5
        \begin{itemize}
        \item this is the 'core' module
        \item the default build of d3.js includes the core, scale, svg and behavior modules
        \item others include:
            \begin{itemize}
            \item time
            \item geo
            \item csv
            \end{itemize}
        \end{itemize}
    \end{itemize}
\end{frame}


\begin{frame}
    \frametitle{Exercise-01.html}
    \begin{itemize}
    \item all d3 commands live in a unified d3 namespace
    \item d3 supports CSS3 notation, i.e. can select by:
        \begin{itemize}
        \item can select by tag ("div")
        \item class (".awesome")
        \item unique identifier ("#foo")
        \item containment ("parent child")
        \item selectors can be intersected (".this.that" for logical AND) or unioned (".this, .that" for logical OR)
        \item attribute ("[color=red]")
        \end{itemize}
    \item note the difference between d3.select and d3.selectAll
    \item text is an "operator"
        \begin{itemize}
            \item operators can both get or set
            \item other operators include: attr, style, html, classed, property
        \end{itemize}
    \item notice the method chaining has already begun
    \item elements can be accessed directly
        \begin{itemize}
        \item (e.g., selection[0][0])
        \item through the each call
        \end{itemize}
    \end{itemize}
\end{frame}


\begin{frame}
    \frametitle{Exercise-01.html}
    \begin{itemize}
        \begin{itemize}
        \end{itemize}
                \item d3.select("") $\approx$ \$("") $\approx$ jQuery("")

                \item Operators can get or set attr, class, style, property, html and text
                \item attr - 
                \item By default, D3 supports svg, xhtml, xlink, xml and xmlns namespaces. Additional namespaces can be registered by adding to d3.ns.prefix.
                \item class: toggling of css classes
                \item style: sets the css style property, can be run w/ priority levels
                \item property: 
                \item can be set as either constants or as functions
                \item Cascading/callbacks
                When used to set document content, the operators return the current selection, so you can chain multiple operators together in a concise statement.
            \end{itemize}
        \end{itemize}
    \end{block}
\end{frame}


\subsection{Circle and Text}
- change the color of the drop-shadow

\begin{frame}
    \frametitle{Tutorials}
    \begin{block}{exercise-03.html}
        \begin{itemize}
            \pause
            \item A circle with text
            \pause
            \item Concepts
            \begin{itemize}
                \pause
                \item Have to touch on CSS styling a little bit
                note that the css selector notation in the style section is 
                the same as that used in the d3.select("") command
                \pause
                \item SVG elements, reference the SVG spec for relevant attributes
                \item think of an SVG element as the canvas that we're going to draw on

                \item width and height could be related to the width and height of the 
                \item window in order to be self-adjusting
                \pause
                \item Appending is fairly self-explanatory
                \pause
                \item Transforms, coordinate transform to make it easier to think of
                \item Note that the origin is in the top-left corner and that positive y
                      is down.
                \item Jumping ahead, I will commonly account for this, not with a transform, but
                        with an appropriate scale, which we will get to shortly
                \item Note: a side effect of this transform is that by not setting the x,y variables
                        on the circle... it defaults to zero.  This zero is now the point that is
                        transformed to the middle of the page though.
                \pause
                \item Namespaces, explain that svg:svg <-- first one is a namespace, second
                one is the element itself
                svg:g is kind of like a div in html... just a bag in which to group other things
                in
                note: you give them uniqueness through class or id
                \pause
                \item Attr, addressed in previous slide
                \pause
                \item Appropriate use of namespace variables
                \item Assign a namespace at any "juncture" in your workflow
                    i.e. if you're about to add circles AND text to your scenegraph... it's
                        probably appropriate to add a name to the state of your scenegraph at
                        that point
                \pause
                \item 
            \end{itemize}
        \end{itemize}
    \end{block}
\end{frame}

exercise-04.html
event listeners

exercise-05.html
- skip tweens and get to data bindings

exercise-06.html
- notice that we're scaling the whole image, 

exercise-07.html
- listen to user events, i.e watch the mouse move

exercise-08.html
- mouse fading events

exercise-09.html
- how to draw a bar graph
- the basics of appending 

\subsection{Bar Chart}

\begin{frame}
    \frametitle{Tutorials}
    \begin{block}{exercise-10.html}
        \begin{itemize}
            \pause
            \item Bar Chart with HTML Elements
            \pause
            \item have to explain scales... 

            \item Concepts
            \begin{itemize}
                \pause
                \item Identity function
                \pause
                \item Functional programming
                \pause
                \item Data binding selections
                \begin{itemize}
                    \pause
                    \item Update
                    \pause
                    \item Enter
                    \pause
                    \item Exit
                \end{itemize}
                \pause
                \item g namespace
            \end{itemize}
        \end{itemize}
    \end{block}
\end{frame}


\subsection{2d Array into HTML Table}

- some subtle behavior
- if I could ask Lauren to help me put this clearly that would probably help

\begin{frame}
    \frametitle{Tutorials}
    \begin{block}{exercise-11.html}
        \pause
        \begin{itemize}
        \pause
        \item 2d Array into HTML Table
        \pause
        \item Concepts
            \begin{itemize}
                \item Styles
                \item Scales
            \end{itemize}
        \end{itemize}
    \end{block}
\end{frame}


\subsection{2d Array into SVG Bar Chart}

he's used rangebands and ordinal scales


\begin{frame}
    \frametitle{Tutorials}
    \begin{block}{exercise-12.html}
        \pause
        \begin{itemize}
        \pause
        \item 2d Array into SVG Bar Chart
        \pause
        \item Concepts
            \begin{itemize}
                \item RangeBands
                \item Linear vs. ordinal scales
            \end{itemize}
        \end{itemize}
    \end{block}
\end{frame}


exercise-13.html
\subsection{Axes Elements}
axes elements are the stars
do a quick little demo of changing the results to show off how flexible it is
\begin{frame}
    \frametitle{Tutorials}
    \begin{block}{exercise-13.html}
        \begin{itemize}
            \pause
            \item Axes Elements
        \end{itemize} 
    \end{block}
\end{frame}


\subsection
address exercise-04.html
event listeners can take many forms
    - i.e. you can listen for different types of events
    - click, mouseover, submit, etc.
    - there's a subtlety of attaching to multiple functions to the same event... 
      i.e. click.foo -> one function, click.bar -> another function
    - d, i, this...  


I didn't bring over d3.time.js nor d3.csv.js --> because I didn't search very thoroughly

\begin{frame}
    \frametitle{Tutorials}
    \begin{block}{Extras}
        \begin{itemize}
            \item css-hover
            \item d3 event-based transition
            \item Transition : just a non-instantaneous transformation
            \begin{itemize}
                \item Duration
                \item Delay
            \end{itemize}
            \item Ease
            \item Interpolate
            \item Tween
            \item call and each for control flow 
        \end{itemize}
    \end{block}
\end{frame}

%% exercise 14, 15, 16 are largely a wash for now

\section{Conclusion}

\begin{frame}
    \frametitle{Conclusion}
    \begin{itemize}
        \item Blah
    \end{itemize}
\end{frame}

\end{document}

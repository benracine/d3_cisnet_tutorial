%  d3 Tutorial CISNET Programmers Breakout November 2, 2011
%  Author: Ben Racine
%  Date:   November 2, 2011

\documentclass{beamer}

\mode<presentation>
{
    \usetheme{Warsaw}
    \setbeamercovered{transparent}
}

\setbeamercolor{math text}{fg=red!50!black}

\usepackage{amsmath,amssymb}
\usepackage{colortbl}
\usepackage[english]{babel}
\usepackage{fancybox, stmaryrd}
\usepackage[latin1]{inputenc}
\usepackage{lmodern}
\usepackage{pgf, pgfnodes}
\usepackage[T1]{fontenc} 
\usepackage{times}
\usepackage{verbatim}

\definecolor{links}{HTML}{2A1B81}
\hypersetup{colorlinks,linkcolor=,urlcolor=links}


\title{PBO Workshop} 
\subtitle{Data-Driven Documents With d3} 
\author{Ben Racine \inst{1} }
\institute{\inst{1} Cornerstone Systems NW }
\date{November 2, 2011}
\subject{Data Visualization}


\begin{document}


\begin{frame}
    \frametitle{}
    \titlepage
\end{frame}


\begin{frame}
    \scriptsize{
        \frametitle{Outline}
        \tableofcontents[pausesections]
    }
\end{frame}


\section{Introduction}

\subsection{Attendee Introduction}

\begin{frame}
    \frametitle{Introductions}
    \pause
    \begin{block}{Favorite Tools?} 
    \begin{itemize}
        \item Database and/or spreadsheet tools
        \item DSL and/or general programming languages
        \item Visualization tools
        \end{itemize}
    \end{block}
    \pause
    \begin{block}{Any web developers?}
        \begin{itemize}
        \item Javascript experience?
        \end{itemize}
    \end{block}
\end{frame}


\subsection{Browser Poll}

\begin{frame}
    \frametitle{Introductions}
    \begin{block}{Browser Poll}
        \begin{itemize}
        \pause
        \item Chrome
        \pause
        \item Firefox 3+
        \pause
        \item IE9
        \pause
        \item Safari
        \pause
        \item None of the above?
        \end{itemize}
    \end{block}
\end{frame}


\section{Background}

\begin{frame}
    \frametitle{Background}
    \begin{block}{jQuery $+$ Protovis $\approx$ d3}
        \begin{itemize}
        \pause
        \item Any jQuery experience?
        \pause
            \begin{itemize}
            \item D3 is similar, but can also target the SVG (xml-like image format)
        \pause
            \item They both do some fancy functional programming to make it possible 
                  for us to declaratively (and efficiently) reach into the dom tree.
            \end{itemize}
        \pause
        \item Any Protovis exposure by any chance?
       \end{itemize}
    \end{block}
\end{frame}


\section{Resources}

\begin{frame}
    \frametitle{Resources}
        \pause
        \begin{itemize}
        \item \href{http://mbostock.github.com/d3/}{\underline{Github}: http://mbostock.github.com/d3/}
            \begin{itemize}
            \item \href{https://github.com/mbostock/d3/wiki/API-Reference}{\underline{API}: https://github.com/mbostock/d3/wiki/API-Reference}
            \item \href{http://mbostock.github.com/d3/ex/}{\underline{Examples}: http://mbostock.github.com/d3/ex/}
            \item \href{https://github.com/mbostock/d3.git}{\underline{Source}: https://github.com/mbostock/d3.git}
            \end{itemize}
        \pause
        \item \href{http://groups.google.com/group/d3-js?pli=1}{Google group}
        \pause
        \item \href{http://www.w3.org/TR/SVG/}{SVG Specification (v1.1)}
        \pause
        \item Twitter: $@i3enhamin, @mbostock$
        \end{itemize}
\end{frame}



\section{Installation}

\begin{frame}[fragile]
    \frametitle{Introductions}
    \begin{block}{Decision Tree}
        \tiny{
        \begin{verbatim}
if you are comfortable developing in the browser
   if you are a git user
       git clone git@github.com:benracine/d3_cisnet_tutorial.git
   else
       https://github.com/benracine/d3_cisnet_tutorial/downloads
   end
else
    jsfiddle links
end
        \end{verbatim}
        }
    \end{block}
%   \footnote{Recommended to serve up on localhost on Chrome, i.e. python -m SimpleHTTPServer 8000}
    \footnote{jsfiddle examples still need to be done}
\end{frame}


\section{Tutorials}


\begin{frame}
    \frametitle{Ready to Get Started?}
    \begin{block}{Canonical Test}
        \begin{itemize}
        \pause
        \item Navigate to an exercise file
        \pause
        \item Open up your browser's web developer tools
        \pause
        \item Go to console
        \pause
        \item Enter d3 and you should see 'object' in the response
        \end{itemize}
    \end{block}
\end{frame}


\subsection{Hello World}

\begin{frame}
    \frametitle{Tutorials}
    \begin{block}{exercise-01.html}
        \begin{itemize}
            \pause
            \item  Hello world only using HTML (i.e. no SVG)
            \pause
            \item  Concepts
            \begin{itemize}
                \pause
    			\item Sourcing d3 file 
                \pause
                \item Unified d3 namepace
                \pause
                \item css3 selector notation
                \pause
                \begin{itemize}
                    \item Element, class, id, attribute, descendants, etc.
                \end{itemize}
                \pause
                \item d3.select("") $\approx$ \$("") $\approx$ jQuery("")
                \pause
                \item Cascading/callbacks
            \end{itemize}
        \end{itemize}
    \end{block}
\end{frame}


\subsection{Circle and Text}

\begin{frame}
    \frametitle{Tutorials}
    \begin{block}{exercise-03.html}
        \begin{itemize}
            \pause
            \item A circle with text
            \pause
            \item Concepts
            \begin{itemize}
                \pause
                \item SVG elements
                \pause
                \item Appending
                \pause
                \item Tranforms
                \pause
                \item Namespaces
                \pause
                \item Attr
                \pause
                \item Appropriate use of namespace variables
            \end{itemize}
        \end{itemize}
    \end{block}
\end{frame}


\subsection{Bar Chart}

\begin{frame}
    \frametitle{Tutorials}
    \begin{block}{exercise-10.html}
        \begin{itemize}
            \pause
            \item Bar Chart with HTML Elements
            \pause
            \item Concepts
            \begin{itemize}
                \pause
                \item Identity function
                \pause
                \item Functional programming
                \pause
                \item Data binding selections
                \begin{itemize}
                    \pause
                    \item Update
                    \pause
                    \item Enter
                    \pause
                    \item Exit
                \end{itemize}
                \pause
                \item g namespace
            \end{itemize}
        \end{itemize}
    \end{block}
\end{frame}


\subsection{2d Array into HTML Table}

\begin{frame}
    \frametitle{Tutorials}
    \begin{block}{exercise-11.html}
        \pause
        \begin{itemize}
        \pause
        \item 2d Array into HTML Table
        \pause
        \item Concepts
            \begin{itemize}
                \item Styles
                \item Scales
            \end{itemize}
        \end{itemize}
    \end{block}
\end{frame}


\subsection{2d Array into SVG Bar Chart}

\begin{frame}
    \frametitle{Tutorials}
    \begin{block}{exercise-12.html}
        \pause
        \begin{itemize}
        \pause
        \item 2d Array into SVG Bar Chart
        \pause
        \item Concepts
            \begin{itemize}
                \item RangeBands
                \item Linear vs. ordinal scales
            \end{itemize}
        \end{itemize}
    \end{block}
\end{frame}


\subsection{Axes Elements}

\begin{frame}
    \frametitle{Tutorials}
    \begin{block}{exercise-13.html}
        \begin{itemize}
            \pause
            \item Axes Elements
        \end{itemize} 
    \end{block}
\end{frame}


\begin{frame}
    \frametitle{Tutorials}
    \begin{block}{Extras}
        \begin{itemize}
            \item css-hover
            \item d3 event-based transition
            \item Transition : just a non-instantaneous transformation
            \begin{itemize}
                \item Duration
                \item Delay
            \end{itemize}
            \item Ease
            \item Interpolate
            \item Tween
        \end{itemize}
    \end{block}
\end{frame}


\section{Conclusion}

\begin{frame}
    \frametitle{Conclusion}
    \begin{itemize}
        \item Blah
    \end{itemize}
\end{frame}

\end{document}

%  d3 Tutorial CISNET Programmers Breakout November 2, 2011
%  Author: Ben Racine
%  Date:   November 2, 2011
%

\documentclass{beamer}
\mode<presentation>
{
    \usetheme{Warsaw}
    \setbeamercovered{transparent}
}

\setbeamercolor{math text}{fg=red!50!black}

\usepackage{amsmath,amssymb}
\usepackage{colortbl}
\usepackage[english]{babel}
\usepackage{fancybox, stmaryrd}
\usepackage[latin1]{inputenc}
\usepackage{lmodern}
\usepackage{pgf, pgfnodes}
\usepackage[T1]{fontenc} 
\usepackage{times}
\usepackage{verbatim}

\definecolor{links}{HTML}{2A1B81}
\hypersetup{colorlinks,linkcolor=,urlcolor=links}


\title{PBO Workshop} 
\subtitle{Data-Driven Documents With d3} 
\author{Ben Racine \inst{1} }
\institute{\inst{1} Cornerstone Systems NW }
\date{November 2, 2011}
\subject{Data Visualization}

\begin{document}


\begin{frame}
    \frametitle{}
    \titlepage
\end{frame}


%\begin{frame}
%    \scriptsize{
%        \frametitle{Outline}
%        \tableofcontents[pausesections]
%    }
%\end{frame}


\section{d3 Wow}

%\begin{frame}
    \frametitle{d3 Wow
    \begin{itemize}
    \item Solves the crux visualization
        \begin{itemize}
        \item Scales
            \begin{itemize}
            \item Quantitative scales: Linear, log, power, discrete/continuous, 
            \item Ordinal scales
            \item Both can be mapped to colors
            \item invert() allows for two way scaling
            \item User-defined
            \end{itemize}
        \item Intelligently binding data to style elements
        \end{itemize}
    \item Interactively working at different levels of detail
        \begin{itemize}
        \item Circle packing
        \item Radial
        \item Icicle Tree
        \end{itemize}
    \item Makes geo-located data a snap
        \begin{itemize}
        \item Our little side project for Startup Weekend
        \end{itemize}
    \item Flexible scales, axis elements, lines, areas and shapes enable the usual suspects
        \begin{itemize}
        \item Bar/column 
        \item Pie 
        \item Scatter 
        \item Line 
        \end{itemize}
    \end{itemize}
%\end{frame}

\end{document}

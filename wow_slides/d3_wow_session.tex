%  d3 Tutorial CISNET Programmers Breakout November 2, 2011
%  Author: Ben Racine
%  Date:   November 2, 2011
% 

\documentclass{beamer}
\mode<presentation>
{
    \usetheme{Warsaw}
    \setbeamercovered{transparent}
}

\setbeamercolor{math text}{fg=red!50!black}

\usepackage{amsmath,amssymb}
\usepackage{colortbl}
\usepackage[english]{babel}
\usepackage{fancybox, stmaryrd}
\usepackage[latin1]{inputenc}
\usepackage{lmodern}
\usepackage{pgf, pgfnodes}
\usepackage[T1]{fontenc} 
\usepackage{times}
\usepackage{verbatim}

\definecolor{links}{HTML}{2A1B81}
\hypersetup{colorlinks,linkcolor=,urlcolor=links}

\title{PBO Workshop} 
\subtitle{Getting Excited About Data Visualization With d3} 
\author{Ben Racine \inst{1} }
\institute{\inst{1} Cornerstone Systems NW }
\date{November 2, 2011}
\subject{Data Visualization}





 % \begin{frame}
 %    \scriptsize{
 %        \frametitle{Outline}
 %        \tableofcontents[pausesections]
 %    }
 % \end{frame}


\begin{document}


\begin{frame}
    \frametitle{}
    \titlepage
\end{frame}



\section{The Philosophy of d3}



\begin{frame}
\frametitle{The Philosophy of d3}
\begin{itemize}
\pause
\item d3 isn't "just" a charting library
\pause
\item It is a tool for building dynamic and interactive visualizations
\pause
\item But, it does come with (a quickly growing number of) examples and layouts
\end{itemize}
\end{frame}



\begin{frame}
\frametitle{The Philosophy of d3}
\begin{itemize}
\pause
\item d3 is small and doesn't pollute the global namespace
\pause
\item d3 honors modern web standards instead of reinventing them
\pause
\item This empowers developers with reusable knowledge
\pause
\item d3 exposes fine-grain control
\pause
\item d3 does "one" thing and does it well, reminiscent of the UNIX spirit
\end{itemize}
\end{frame}




\section{The Cruxes of Visualization}


\begin{frame}
\frametitle{d3 Solves the Cruxes of Visualization}
\begin{itemize}
\pause
\item Enables the binding of data to styled presentation elements 
    \begin{itemize}
\pause
    \item Provides the ability to update, append and remove these bonds 
\pause
    \item Provides the ability to obtain and update new data by url (formats include text, json, xml, html, csv)
\pause
    \item Provides transitions between different presentation styles in response to user interaction
    \end{itemize}
\end{itemize}
\end{frame}



\begin{frame}
\frametitle{d3 Solves the Cruxes of Visualization}
\begin{itemize}
\pause
\item Provides scales to define the mapping between between data and pixels
\pause
    \begin{itemize}
    \item Quantitative scales for continuous domains (linear, power, log, etc.)
\pause
    \item Ordinal scales for categorical domains
    \end{itemize}
\pause
\item Targets the browser, making it ideal for quickly reaching a wide audience
    \begin{itemize}
\pause
    \item And there is a story for getting to print
    \end{itemize}
\end{itemize}
\end{frame}



\section{Background}



\begin{frame}
\frametitle{Background}
\begin{itemize}
\item d3.js is loosely associated with the \href{http://vis.stanford.edu/}{\underline{Stanford Visualization Group}}
\pause
\item Supersedes the ProtoVis project, so resources found there can be relevant
\pause
\item Mike Bostock is the primary author
\pause
\item Open sourced on GitHub.com
\pause
\item Only a year or two old
\end{itemize}
\end{frame}



\begin{frame}
\frametitle{Background, cont.}
It largely targets the SVG element, which is being increasingly supported across all major browsers:
\pause
    \begin{itemize}
    \item Internet Explorer, 9+
    \item Chrome
    \item Firefox
    \item Safari
    \item iOS
    \item Android, 3.0+
    \item Opera, 9.5+
    \end{itemize}
\pause
\footnote{JavaScript performance has increased dramatically in the last few years}
\end{frame}



\section{Tour}



\begin{frame}
\frametitle{Examples: UK University Statistics}
\begin{itemize}
\item A dynamic linked tree-map and line-chart
    \begin{itemize}
    \item Note the disparity between rising applications and acceptance rates
    \item Show the increasing applications from women relative to men
    \item Show the dramatic increase in 25-39 year old women applications
    \end{itemize}
\end{itemize}
\end{frame}



\begin{frame}
\frametitle{Examples: World Water Resources}
\begin{itemize}
\item Linked scatter, line, and bar charts
    \begin{itemize}
    \item Show the island and desert nations
    \item There should be a navigation map
    \item There should be a tooltip upon mouseover in the upper-right plot
    \item Show Papua New Guinea's low water usage
    \end{itemize}
\end{itemize}
\end{frame}



\begin{frame}
\frametitle{Examples: Anderson's Data of Iris Flowers on the Gaspé Peninsula}
\begin{itemize}
\item The scatterplot matrix visualizations pairwise correlations for multi-dimensional data
\item Selection of subgroups in a scatterplot matrix
\end{itemize}
\end{frame}



\begin{frame}
\frametitle{Examples: Show Reel}
\begin{itemize}
\item Just a gratuitous example of transitions
\end{itemize}
\end{frame}



\begin{frame}
\frametitle{Examples: US Population by Age Over the Years}
\begin{itemize}
\item Note the aging out on the left side of the page
\item This graphic motivates our final exercise this afternoon
\end{itemize}
\end{frame}



\begin{frame}
\frametitle{Examples: Where Does Our Tax Money Go?}
\begin{itemize}
\item Note the differences between \$25k, \$250k, \$2.5 million
\end{itemize}
\end{frame}



\begin{frame}
\frametitle{Examples: Startup Weekend Map}
\begin{itemize}
\item I taught an Architect that had never coded before how to do this in about two days
\end{itemize}
\end{frame}


\begin{frame}
\frametitle{Examples: Circos}
\begin{itemize}
\item A multi-scale chord diagram denoting association between different genetic information
\end{itemize}
\end{frame}


\begin{frame}
\frametitle{Examples: Trulia House Hunting}
\begin{itemize}
\item Note the incredibly high mobile usage in Washington vs. Montana
\item Note the dramatic difference in usage times between the two as well
\end{itemize}
\end{frame}



\end{document}

% \subsection{The Author's Tour}


% \begin{frame}
% \frametitle{The Author's Tour}
% \href{http://mbostock.github.com/d3/talk/20111018/\#0}{A Tour}
% % % \begin{itemize}
% % % \item The author uses "layouts"
% % % \item Association / adjacency representations are addressed with chord layout
% % %     \begin{itemize}
% % %     \item Chord - produce a chord diagram from a matrix of relationships.
% % %     \end{itemize}
% % % \item A whole collection of hierarchical layouts 
% % %     \begin{itemize}
% % %     \item Bundle - apply Holten's hierarchical bundling algorithm to edges.
% % %     \item Cluster - cluster entities into a dendrogram.
% % %     \item Hierarchy - derive a custom hierarchical layout implementation.
% % %     \item Histogram - compute the distribution of data using quantized bins.
% % %     \item Pack - produce a hierarchical layout using recursive circle-packing.
% % %     \item Partition - recursively partition a node tree into a sunburst or icicle.
% % %     \item Stack - compute the baseline for each series in a stacked bar or area chart.
% % %     \item Tree - position a tree of nodes tidily.
% % %     \item Treemap - use recursive spatial subdivision to display a tree of nodes.
% % %     \end{itemize}
% % % \end{itemize}
% \end{frame}



% \begin{frame}
% \frametitle{The Author's Tour Continued}
% \begin{itemize}
% \item Calendar
% \item Time-series (note Mike's other project: \href{https://github.com/square/cube}{"Cube"})
% \item Pan and zoom 
% \item Scale elements (with tick, label, title and location options)
% \item Smooth transitions
% \item Interaction 
% \item Animation
% \item Chloropleth / projections
% \item Force (helpful in solving the non-collision problem)
% \end{itemize}
% \end{frame}



% \subsection{Additional Examples}

% \begin{frame}
% \frametitle{Additional Examples}
% \begin{itemize}
% \item UK University - statistics
% \item Box plot - quintile distribution of a single variable
% \item Bullet - considered a best-practices replacement for gauge charts
% \end{itemize}
% \end{frame}


% \item Histogram - compute the distribution of data using quantized bins.
% \item Bar / column 
% \item Hyperbolic tree
% \item Pie - compute the start and end angles for arcs in a pie or donut chart.
% \item QQ plots - compare two probability distributions by graphing their quantiles against each other
% \item Radial - 
% \item Stack - 
% \item Streamgraph - a generalization of stacked area graphs


% \subsection{My Tour}

% \begin{frame}
% \frametitle{My Tour}
% \begin{itemize}
% \item A side project that I consulted on for the \href{http://www.startupweekendviz.johnmorefield.com/d3/examples/SW/map.html}{Startup Weekend}
% \item ** \href{http://localhost:9000/examples/my_cisnet_demo/navpane.html}{My navpane task}
% \end{itemize}
% \end{frame}




%  d3 Tutorial CISNET Programmers Breakout November 2, 2011
%  Author: Ben Racine
%  Date:   November 2, 2011

\documentclass{beamer}
\mode<presentation>
{
    \usetheme{Warsaw}
    \setbeamercovered{transparent}
}

\setbeamercolor{math text}{fg=red!50!black}

\usepackage{amsmath,amssymb}
\usepackage{colortbl}
\usepackage[english]{babel}
\usepackage{fancybox, stmaryrd}
\usepackage[latin1]{inputenc}
\usepackage{lmodern}
\usepackage{pgf, pgfnodes}
\usepackage[T1]{fontenc} 
\usepackage{times}
\usepackage{verbatim}

\definecolor{links}{HTML}{2A1B81}
\hypersetup{colorlinks,linkcolor=,urlcolor=links}

\title{PBO Workshop} 
\subtitle{Data-Driven Documents With d3} 
\author{Ben Racine \inst{1} }
\institute{\inst{1} Cornerstone Systems NW }
\date{November 2, 2011}
\subject{Data Visualization}


\begin{document}

\begin{frame}
    \frametitle{}
    \titlepage
\end{frame}


% \begin{frame}
%    \scriptsize{
%        \frametitle{Outline}
%        \tableofcontents[pausesections]
%    }
% \end{frame}



\section{Background}

\begin{frame}
\frametitle{Background}
\begin{itemize}
\item d3.js is a small, free JavaScript library for manipulating documents based on data.
\item d3 allows you to bind arbitrary data to a Document Object Model (DOM), and then apply data-driven transformations to the document. 
\item For example:
    \begin{itemize}
    \item You can use d3 to generate a basic HTML table from a matrix of numbers.
    \item You could use the same data to create an interactive SVG bar chart with smooth transitions and interaction.
    \end{itemize}
\item Further:
    \begin{itemize}
    \item SVG is supported across all the modern browsers including
    \item JavaScript performance has recently increased considerably
    \end{itemize}
\end{itemize}
\end{frame}



\begin{frame}
\frametitle{Background (cont.)}
\begin{itemize}
\item d3.js is loosely associated with the \href{http://vis.stanford.edu/}{\underline{Stanford Visualization Group}}
\item Supersedes the ProtoVis project
\item Mike Bostock is the primary author, but it is open source on Github
\item Only a year or two old
\end{itemize}
\end{frame}



\section{Tour}

\subsection{The Author's Tour}

\begin{frame}
\frametitle{The Author's Tour}
\href{http://mbostock.github.com/d3/talk/20111018/#2}{A Tour}
\begin{itemize}
\item Association / adjacency representations via chord diagrams
\item Various hierarchical layouts (bundle, cluster, partition, pack, treemap, hierarchical bar charts)
\item Calendar
\item Time-series 
\item Pan and zoom 
\item Scale elements (with tick, label, title and location options)
\item Smooth transitions
\item Interaction 
\item Animation
\item Chloropleth / projections
\item Force (helpful in solving the non-collision problem)
\end{itemize}
\end{frame}



\subsection{My Tour}

\begin{frame}
\frametitle{My Tour}
\begin{itemize}
\item 1242141265!@#@!#^%!#
\item Still need to clean up this version
\item \href{http://localhost:9000/examples/my_cisnet_demo/navpane.html}{My navpane task}
\item See a side project that I consulted on for the \href{http://www.startupweekendviz.johnmorefield.com/d3/examples/SW/map.html}{Startup Weekend}
\end{itemize}
\end{frame}



\subsection{Repository Examples}

\begin{frame}
\frametitle{Repository Examples}
\begin{itemize}
\item Bar/column 
\item Box plot
\item Bullet
\item Histogram
\item Hyperbolic tree
\item Pie 
\item qq plots
\item Radial
\item Stack
\item Streamgraph
\end{itemize}
\end{frame}



\section{The Philosophy of d3}

\subsection{Why d3?}

\begin{frame}
\frametitle{Why d3?}
\begin{itemize}
\item It is a small and sharp tool 
\item It does "one" thing and does it well
\item Plays well with others
\item Doesn't pollute the global namespace
\end{itemize}
\end{frame}



\subsection{The Cruxes of Visualization}

\begin{frame}
\frametitle{The Cruxes of Visualization}
\begin{block}{Scales}
\begin{itemize}
\item Scales define the mapping between the data and pixel/color space
\item A reverse function can be used to map the other direction too
\item Linear, log, power, discrete/continuous, banded, ordinal and user defined
\end{itemize}
\end{block}
\begin{block}{Data Binding}
\begin{itemize}
\item Detailed control over the ability append, remove, union and difference the two
\end{itemize}
\end{block}
\end{frame}



\subsection{Reuse of Existing Standards}

\begin{frame}
\frametitle{
\begin{itemize}{Reuse of Existing Standards}
\item Takes advantage of existing w3 standards instead of reinventing the wheel
\item Authors can take their new depth of knowledge of modern web standards with them
\end{itemize}
\end{frame}


\end{document}
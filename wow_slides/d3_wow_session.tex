%  d3 Tutorial CISNET Programmers Breakout November 2, 2011
%  Author: Ben Racine
%  Date:   November 2, 2011

\documentclass{beamer}
\mode<presentation>
{
    \usetheme{Warsaw}
    \setbeamercovered{transparent}
}

\setbeamercolor{math text}{fg=red!50!black}

\usepackage{amsmath,amssymb}
\usepackage{colortbl}
\usepackage[english]{babel}
\usepackage{fancybox, stmaryrd}
\usepackage[latin1]{inputenc}
\usepackage{lmodern}
\usepackage{pgf, pgfnodes}
\usepackage[T1]{fontenc} 
\usepackage{times}
\usepackage{verbatim}

\definecolor{links}{HTML}{2A1B81}
\hypersetup{colorlinks,linkcolor=,urlcolor=links}

\title{PBO Workshop} 
\subtitle{Getting Excited About Data-Driven Documents With d3} 
\author{Ben Racine \inst{1} }
\institute{\inst{1} Cornerstone Systems NW }
\date{November 2, 2011}
\subject{Data Visualization}


\begin{document}


\begin{frame}
    \frametitle{}
    \titlepage
\end{frame}


 \begin{frame}
    \scriptsize{
        \frametitle{Outline}
        \tableofcontents[pausesections]
    }
 \end{frame}



\section{If you are eager to get these materials}

\begin{frame}
    \frametitle{If you are eager to obtain everything}
    \begin{itemize}
\pause
    \item Navigate to github.com
\pause
    \item Search for benracine
\pause
    \item This repo should be the first hit, i.e. "d3\_cisnet\_tutorial"
    \end{itemize}
\end{frame}


\section{Background}


\begin{frame}
\frametitle{Background}
\begin{itemize}
\item d3.js is a small, free JavaScript library for manipulating documents based on data.
\item d3 allows you to bind arbitrary data to a Document Object Model (DOM), and then apply data-driven transformations to the document. 
\item For example:
    \begin{itemize}
    \item You can use d3 to generate a basic HTML table from a matrix of numbers.
    \item You could use the same data to create an interactive SVG bar chart with smooth transitions and interaction.
    \end{itemize}
\end{itemize}
\end{frame}



\begin{frame}
\frametitle{Ecosystem}
\begin{itemize}
\item It largely targets the SVG element
\item SVG is being increasingly supported:
    \begin{itemize}
    \item IE, 9+
    \item Chrome
    \item Firefox
    \item Safari
    \item Opera, 9.5+
    \item iOS
    \item Android, 3.0+
    \end{itemize}
\item A few additional JavaScript libraries can be useful for minor data transformations and persistence
    \begin{itemize}
    \item Underscore.js
    \item Backbone.js
    \item JSONSelect.js 
    \item Science.js
    \end{itemize}
\item JavaScript performance has recently increased considerably
\item JavaScript has a beautiful functional language buried under some famous warts
\end{itemize}
\end{frame}



\begin{frame}
\frametitle{Background (cont.)}
\begin{itemize}
\item d3.js is loosely associated with the \href{http://vis.stanford.edu/}{\underline{Stanford Visualization Group}}
\item Supersedes the ProtoVis project
\item Mike Bostock is the primary author, but it is open source on Github.com
\item Only a year or two old
\end{itemize}
\end{frame}



\section{Tour}


\subsection{The Author's Tour}


\begin{frame}
\frametitle{The Author's Tour}
\href{http://mbostock.github.com/d3/talk/20111018/#0}{A Tour}
\begin{itemize}
\item The author uses "layouts"
\item Association / adjacency representations are addressed with chord layout
    \begin{itemize}
    \item Chord - produce a chord diagram from a matrix of relationships.
    \end{itemize}
\item A whole collection of hierarchical layouts 
    \begin{itemize}
    \item Bundle - apply Holten's hierarchical bundling algorithm to edges.
    \item Cluster - cluster entities into a dendrogram.
    \item Hierarchy - derive a custom hierarchical layout implementation.
    \item Histogram - compute the distribution of data using quantized bins.
    \item Pack - produce a hierarchical layout using recursive circle-packing.
    \item Partition - recursively partition a node tree into a sunburst or icicle.
    \item Stack - compute the baseline for each series in a stacked bar or area chart.
    \item Tree - position a tree of nodes tidily.
    \item Treemap - use recursive spatial subdivision to display a tree of nodes.
    \end{itemize}
\end{itemize}
\end{frame}



\begin{frame}
\frametitle{The Author's Tour Continued}
\begin{itemize}
\item Calendar
\item Time-series (note Mike's other project: \href{https://github.com/square/cube}{"Cube"})
\item Pan and zoom 
\item Scale elements (with tick, label, title and location options)
\item Smooth transitions
\item Interaction 
\item Animation
\item Chloropleth / projections
\item Force (helpful in solving the non-collision problem)
\end{itemize}
\end{frame}



\subsection{Repository Examples}

\begin{frame}
\frametitle{Repository Examples}
\begin{itemize}
\item Bar / column 
\item Box plot - quintile distribution of a single variable
\item Bullet - considered a best-practices replacement for gauge charts
\item Histogram - compute the distribution of data using quantized bins.
\item Hyperbolic tree
\item Pie - compute the start and end angles for arcs in a pie or donut chart.
\item QQ plots - compare two probability distributions by graphing their quantiles against each other
\item Radial - 
\item Stack - 
\item Streamgraph - a generalization of stacked area graphs
\end{itemize}
\end{frame}



\subsection{My Tour}

\begin{frame}
\frametitle{My Tour}
\begin{itemize}
\item A side project that I consulted on for the \href{http://www.startupweekendviz.johnmorefield.com/d3/examples/SW/map.html}{Startup Weekend}
\item ** \href{http://localhost:9000/examples/my_cisnet_demo/navpane.html}{My navpane task}
\end{itemize}
\end{frame}



\section{The Philosophy of d3}

\subsection{Why d3?}

\begin{frame}
\frametitle{Why d3?}
\begin{itemize}
\item It is a small and sharp tool 
\item Kind of in the Unix philosophy, it does "one" thing and does it well
\begin{itemize}
\item i.e. It solves the cruxes of visualization
\end{itemize}
\item Plays well with others
\begin{itemize}
\item i.e. doesn't pollute the global namespace
\end{itemize}
\end{itemize}
\end{frame}



\subsection{The Cruxes of Visualization}

\begin{frame}
\frametitle{The Cruxes of Visualization}
\begin{block}{Scales}
\begin{itemize}
\item Data access by url
\begin{itemize}
\item text
\item json
\item xml
\item html
\item csv
\end{itemize}
\item Scales define the mapping between the data and pixel/color space
\item A reverse function can be used to map the other direction too
\item Linear, log, power, discrete/continuous, banded, ordinal and user defined
\end{itemize}
\end{block}
\begin{block}{Data Binding}
\begin{itemize}
\item Detailed control over the ability append, remove, union and difference the two
\end{itemize}
\end{block}
\end{frame}



\subsection{Reuse of Existing Standards}

\begin{frame}
\frametitle{
\begin{itemize}{Reuse of Existing Standards}
\item Takes advantage of existing w3 standards instead of reinventing the wheel
\item Authors can take their new depth of knowledge of modern web standards with them
\end{itemize}
\end{frame}



\end{document}
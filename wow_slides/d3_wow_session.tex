%  d3 Tutorial CISNET Programmers Breakout November 2, 2011
%  Author: Ben Racine
%  Date:   November 2, 2011
% 

\documentclass{beamer}
\mode<presentation>
{
    \usetheme{Warsaw}
    \setbeamercovered{transparent}
}

\setbeamercolor{math text}{fg=red!50!black}

\usepackage{amsmath,amssymb}
\usepackage{colortbl}
\usepackage[english]{babel}
\usepackage{fancybox, stmaryrd}
\usepackage[latin1]{inputenc}
\usepackage{lmodern}
\usepackage{pgf, pgfnodes}
\usepackage[T1]{fontenc} 
\usepackage{times}
\usepackage{verbatim}

\definecolor{links}{HTML}{2A1B81}
\hypersetup{colorlinks,linkcolor=,urlcolor=links}

\title{PBO Workshop} 
\subtitle{Getting Excited About Data-Driven Documents With d3} 
\author{Ben Racine \inst{1} }
\institute{\inst{1} Cornerstone Systems NW }
\date{November 2, 2011}
\subject{Data Visualization}


\begin{document}


\begin{frame}
    \frametitle{}
    \titlepage
\end{frame}


 \begin{frame}
    \scriptsize{
        \frametitle{Outline}
        \tableofcontents[pausesections]
    }
 \end{frame}



\section{The Philosophy of d3}


\subsection{The Philosophy of d3}

\begin{frame}
\frametitle{The Philosophy of d3}
\begin{itemize}
\item d3 isn't just another charting library
    \begin{itemize}
    \item Although it does happen to come with a lot of prepackaged layouts and examples
    \end{itemize}
\item d3 is a very small and sharp tool that plays nicely with others
\item d3 doesn't pollute the global namespace
\item d3 honors and exposes the user to web standards instead of reinventing them
    \begin{itemize}
    \item Authors can take their new depth of knowledge of modern web standards with them
    \end{itemize}
\item d3 does "one" thing and does it well, reminiscent of the UNIX spirit
\end{itemize}
\end{frame}


\subsection{The Cruxes of Visualization}

\begin{frame}
\frametitle{d3 Solves the Cruxes of Visualization}
Well, to be accurate it does more than one thing, it:
\begin{itemize}
\item Binds data and styled presentation elements 
    \begin{itemize}
    \item Provides control over the ability to update, append and remove these bonds
    \end{itemize}
\item Provides scales to define the mapping back and forth between between data and pixel space
    \begin{itemize}
    \item These scales include linear, log, power, discrete/continuous, banded, ordinal, etc.
    \end{itemize}
\item Applies data-driven dynamic transformations to the document
    \begin{itemize}
    \item Locate data anywhere by url (formats include text, json, xml, html, csv)
    \end{itemize}
\item Transitions between different style representations
\item Targets the browser, making it ideal for quickly reaching a wide audience
    \begin{itemize}
    \item And is there a story for getting to paper from there
    \end{itemize}
\end{itemize}
\end{frame}



\subsection{Background}


\begin{frame}
\frametitle{Background}
\begin{itemize}
\item d3.js is loosely associated with the \href{http://vis.stanford.edu/}{\underline{Stanford Visualization Group}}
\item Supersedes the ProtoVis project, so resources found there can be relevant
\item Mike Bostock is the primary author
\item Open sourced on the highly active GitHub.com
\item Only a year or two old
\end{itemize}
\end{frame}


\begin{frame}
\frametitle{Background, cont.}
It largely targets the SVG element, which is being increasingly supported across all major browsers:
    \begin{itemize}
    \item Internet Explorer, 9+
    \item Chrome
    \item Firefox
    \item Safari
    \item iOS
    \item Android, 3.0+
    \item Opera, 9.5+
    \end{itemize}
It should be noted that JavaScript has had huge investments in increaing its performance recently
\end{frame}



\section{Tour}


\subsection{The Author's Tour}


\begin{frame}
\frametitle{The Author's Tour}
\href{http://mbostock.github.com/d3/talk/20111018/\#0}{A Tour}
\begin{itemize}
\item The author uses "layouts"
\item Association / adjacency representations are addressed with chord layout
    \begin{itemize}
    \item Chord - produce a chord diagram from a matrix of relationships.
    \end{itemize}
\item A whole collection of hierarchical layouts 
    \begin{itemize}
    \item Bundle - apply Holten's hierarchical bundling algorithm to edges.
    \item Cluster - cluster entities into a dendrogram.
    \item Hierarchy - derive a custom hierarchical layout implementation.
    \item Histogram - compute the distribution of data using quantized bins.
    \item Pack - produce a hierarchical layout using recursive circle-packing.
    \item Partition - recursively partition a node tree into a sunburst or icicle.
    \item Stack - compute the baseline for each series in a stacked bar or area chart.
    \item Tree - position a tree of nodes tidily.
    \item Treemap - use recursive spatial subdivision to display a tree of nodes.
    \end{itemize}
\end{itemize}
\end{frame}



\begin{frame}
\frametitle{The Author's Tour Continued}
\begin{itemize}
\item Calendar
\item Time-series (note Mike's other project: \href{https://github.com/square/cube}{"Cube"})
\item Pan and zoom 
\item Scale elements (with tick, label, title and location options)
\item Smooth transitions
\item Interaction 
\item Animation
\item Chloropleth / projections
\item Force (helpful in solving the non-collision problem)
\end{itemize}
\end{frame}



\subsection{Repository Examples}

\begin{frame}
\frametitle{Repository Examples}
\begin{itemize}
\item Bar / column 
\item Box plot - quintile distribution of a single variable
\item Bullet - considered a best-practices replacement for gauge charts
\item Histogram - compute the distribution of data using quantized bins.
\item Hyperbolic tree
\item Pie - compute the start and end angles for arcs in a pie or donut chart.
\item QQ plots - compare two probability distributions by graphing their quantiles against each other
\item Radial - 
\item Stack - 
\item Streamgraph - a generalization of stacked area graphs
\end{itemize}
\end{frame}



\subsection{My Tour}

\begin{frame}
\frametitle{My Tour}
\begin{itemize}
\item A side project that I consulted on for the \href{http://www.startupweekendviz.johnmorefield.com/d3/examples/SW/map.html}{Startup Weekend}
\item ** \href{http://localhost:9000/examples/my_cisnet_demo/navpane.html}{My navpane task}
\end{itemize}
\end{frame}



\end{document}
